\begin{resumo}
O desenvolvimento de jogos eletrônicos é uma área multidisciplinar que exige a integração de diversas práticas e conhecimentos, como design, música, psicologia, programação e gestão de projetos. Este trabalho tem como objetivo o desenvolvimento de um jogo digital estilo Tron. O processo de desenvolvimento foi estruturado utilizando metodologias como Scrum, para organizar as fases do projeto e gerenciar o tempo e as entregas, MDA (\textit{Mechanics-Dynamics-Aesthetics}) para guiar o design, alinhando mecânicas, dinâmicas e estética com a experiência desejada, além de Prototipagem Rápida e Design Iterativo para testar e ajustar o jogo em desenvolvimento. O trabalho aplica conhecimentos de engenharia de software, incluindo Metodologias Ágeis e estratégias de modelagem, com o intuito de oferecer uma abordagem prática e organizada para o desenvolvimento de jogos digitais.

\vspace{\onelineskip}

\noindent
\textbf{Palavras-chave}: Jogo digital. Game Design. Scrum. MDA. Tron.
\end{resumo}
