\chapter*[Introdução]{Introdução}
\addcontentsline{toc}{chapter}{Introdução}
\markboth{Introdução}{Introdução}

Para compreender os objetivos e as implicações deste projeto, é fundamental entender o que caracteriza um jogo. Um jogo pode ser definido como uma atividade estruturada, guiada por regras e objetivos, que envolve desafios e, muitas vezes, interação social. Segundo Johan Huizinga (2000), em sua obra \textit{Homo Ludens}, o jogo é uma manifestação cultural que transcende o mero entretenimento, sendo parte essencial do desenvolvimento humano \cite{huizinga2000}. Jogos têm a capacidade de estimular criatividade, habilidades de resolução de problemas e o pensamento estratégico, promovendo uma experiência lúdica que envolve engajamento emocional e intelectual.

Ao incorporar tecnologia computacional em um jogo, surge o conceito de jogo eletrônico, uma mídia interativa que combina elementos de narrativa, gráficos, áudio e jogabilidade.

Os jogos eletrônicos possuem diversas categorias, e uma delas é representada pelos jogos \textit{Tron}. Esses jogos têm origem no filme “Tron: Uma Odisseia Eletrônica”, da Disney, lançado em 1982, que foi pioneiro em misturar computação gráfica avançada com \textit{live-action} \cite{lawrence2002}. A franquia é ambientada em um universo digital chamado "Grade" (ou \textit{Grid}), onde os personagens interagem com programas e sistemas como se fossem pessoas. Um dos principais elementos desse gênero são os “\textit{Light Cycles}”, corridas em que os jogadores deixam rastros luminosos que funcionam como barreiras mortais. O objetivo é "trancar" os oponentes em um espaço limitado.

Considerando os conceitos e os aspectos citados, junto com os conhecimentos adquiridos no curso de Engenharia de Software, a proposta deste trabalho é desenvolver o software de um jogo digital utilizando elementos dos \textit{Light Cycles} do gênero \textit{Tron}, empregando a Godot, uma plataforma de código aberto para desenvolvimento de jogos 2D e 3D, que oferece uma interface visual baseada em hierarquia de nós, além de suporte a linguagens como GDScript e C\#. Sua escolha se deve à acessibilidade, leveza e ampla documentação disponível \cite{godot}. Além disso, busca-se apresentar a metodologia utilizada no desenvolvimento do jogo digital, com a finalidade de servir como objeto de estudo e aprendizagem.

\section*{Objetivo Geral}
\addcontentsline{toc}{section}{Objetivo Geral}

Desenvolver um jogo digital inspirado nos \textit{Light Cycles} do universo \textit{Tron}, com o objetivo de criar uma versão jogável que possa servir como ferramenta de estudo e análise no contexto do desenvolvimento de jogos digitais. A partir disso, serão aplicados conceitos de Engenharia de Software e metodologias de desenvolvimento ágil, utilizando a engine Godot como plataforma base.

\section*{Objetivos Específicos}
\addcontentsline{toc}{section}{Objetivos Específicos}

\begin{itemize}
  \item Pesquisar e analisar as características dos jogos \textit{Tron} e suas mecânicas principais, com foco nos \textit{Light Cycles}.
  \item Implementar o jogo digital empregando a engine Godot, explorando recursos de narrativa, gráficos, áudio e jogabilidade.
  \item Adotar uma metodologia de desenvolvimento baseada em Scrum e MDA (Mecânica, Dinâmica e Estética), documentando cada etapa do processo.
  \item Integrar os conceitos de Engenharia de Software, como levantamento de requisitos, modelagem e testes, ao ciclo de desenvolvimento do jogo.
  \item Realizar testes simples de caixa-preta e exploração com foco na identificação e correção de erros, garantindo o funcionamento básico e a jogabilidade do sistema.
  \item Apresentar o jogo e a metodologia aplicada como uma ferramenta de estudo para auxiliar iniciantes no desenvolvimento de jogos digitais.
\end{itemize}

\section*{Estrutura do Trabalho}
\addcontentsline{toc}{section}{ Estrutura do Trabalho}

Este trabalho está organizado em oito capítulos, cada um dividido em seções para melhor estruturação e clareza. A divisão seguiu as diretrizes fornecidas pelos professores da FCTE – Faculdade de Ciências e Tecnologias em Engenharia da Universidade de Brasília, conforme o Guia para a Elaboração de Trabalhos de Conclusão de Curso em Engenharia de Software, elaborado pelo Prof. Dr. George Marsicano Corrêa, além do template LaTeX fornecido pelo Prof. Dr. Edson Júnior e das orientações do Prof. Dr. Ricardo Matos Chaim, meu orientador.

O primeiro capítulo apresenta a introdução, oferecendo uma visão geral do trabalho, contextualizando a ideia principal do projeto e detalhando seus objetivos gerais e específicos. Além disso, descreve a estrutura do documento, facilitando a compreensão do leitor.

O segundo capítulo aborda a fundamentação teórica, reunindo os principais conceitos e áreas de conhecimento essenciais tanto para o desenvolvimento de software quanto para a criação de jogos. Esses fundamentos foram aplicados ao longo do projeto para garantir um melhor embasamento e compreensão do tema. Além disso, o capítulo apresenta as ferramentas utilizadas no desenvolvimento do jogo, destacando seus papéis no processo.

O terceiro capítulo apresenta as ferramentas utilizadas no trabalho e seus respectivos usos.

O quarto capítulo apresenta o processo de execução da pesquisa, detalhando as fases e atividades realizadas. São descritas as etapas desde a revisão da literatura até a aplicação e avaliação das técnicas utilizadas, destacando os dados coletados, as metodologias empregadas e a análise dos resultados.

O quinto capítulo apresenta as especificações do jogo, detalhando o processo criativo e as decisões tomadas durante o desenvolvimento. Além disso, conceitos e ideias são explorados com base em pesquisas, análises de materiais e etapas do processo criativo.

O sexto capítulo apresenta os resultados obtidos ao longo do desenvolvimento do jogo. Nele, são discutidas as principais lições aprendidas durante o processo, abordando desafios enfrentados, soluções adotadas e insights adquiridos. Além disso, o capítulo inclui demonstrações do jogo, destacando suas principais funcionalidades e o impacto das decisões tomadas ao longo do projeto.

O sétimo capítulo apresenta a conclusão do projeto, reunindo uma reflexão sobre todo o processo de desenvolvimento. Nele, é exposta minha opinião geral, uma autoavaliação do trabalho realizado e considerações sobre os desafios enfrentados e as soluções adotadas. Além disso, discuto as contribuições do projeto, possíveis melhorias e sugestões para trabalhos futuros.

Por fim, os últimos capítulos apresentam os apêndices e o referencial bibliográfico utilizado na elaboração do trabalho.
