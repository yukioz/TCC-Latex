\chapter[Conclusões Gerais]{Conclusões Gerais}

O trabalho teve como proposta o desenvolvimento de um jogo digital inspirado nos \textit{Light Cycles} do universo Tron, servindo simultaneamente como exercício prático de aplicação de conceitos da Engenharia de Software e como objeto de estudo sobre o processo de criação de jogos digitais. A condução do projeto foi estruturada por meio de Metodologias Ágeis, com destaque para o uso adaptado do framework \textit{Scrum}, que possibilitou organizar as etapas do desenvolvimento de forma iterativa e flexível, respeitando os limites temporais e o escopo previamente definido.

A utilização do modelo MDA (\textit{Mechanics–Dynamics–Aesthetics}) na primeira etapa do trabalho contribuiu para orientar as decisões de design de forma estruturada, promovendo o alinhamento entre os componentes técnicos do jogo e a experiência de jogo pretendida. As mecânicas implementadas, como a geração de rastro, colisões e poderes temporários, foram integradas com foco em proporcionar dinâmicas de disputa rápida e interação estratégica entre jogadores. Ainda que alguns ajustes tenham sido necessários ao longo do desenvolvimento, a estrutura teórica do MDA mostrou-se útil para manter a coerência entre intenção e implementação.

Durante a fase prática, a escolha da Godot Engine atendeu bem às necessidades do projeto, especialmente por seu suporte nativo ao desenvolvimento 2D, sua estrutura baseada em nós hierárquicos e pela facilidade da linguagem \texttt{GDScript}. No entanto, nem sempre os comportamentos da Engine foram intuitivos, o que demandou muitos testes e adaptações na lógica do jogo. Ainda assim, a experiência geral com a ferramenta foi positiva, e a curva de aprendizagem, inicialmente lenta, tornou-se significativamente mais fluida a partir do meio do projeto.

A prática do desenvolvimento proporcionou aprendizado técnico relevante, como o uso de instanciamento dinâmico, gerenciamento de colisões com \texttt{Area2D}, manipulação de grupos e sinais para controle de comportamento, construção de HUD com \texttt{CanvasLayer} e organização modular da lógica de jogo. Além disso, a necessidade constante de testar, depurar e refatorar consolidou competências relacionadas à modelagem, resolução de problemas e tomada de decisão em cenários práticos.

Considerando o escopo proposto, os resultados obtidos demonstram a viabilidade de aplicar fundamentos da Engenharia de Software ao desenvolvimento de jogos digitais com uma abordagem acessível e orientada à aprendizagem. O jogo desenvolvido, mesmo com limitações pontuais, funciona como um protótipo jogável e representa uma base concreta para futuras melhorias ou expansões. Dessa forma, o trabalho cumpre seu papel como objeto de estudo prático e formativo, oferecendo subsídios tanto para o desenvolvimento técnico quanto para a reflexão sobre metodologias aplicadas à criação de jogos.

Por fim, é importante reconhecer que o ciclo de desenvolvimento de um jogo vai além da construção de sua versão jogável. Etapas como empacotamento e exportação multiplataforma, publicação em lojas digitais, monetização, questões de licenciamento e direitos autorais extrapolam o escopo deste trabalho, mas fazem parte do ecossistema do desenvolvimento de jogos e representam oportunidades futuras de aprendizado. No total, foram gastas 291 horas para desenvolver o jogo, marcadas por registros do tempo de uso da ferramenta de controle de sessões.


\section{Lições aprendidas e recomendações}

Durante o desenvolvimento do projeto, algumas lições importantes foram aprendidas, especialmente em relação ao planejamento, à abordagem técnica e ao gerenciamento de riscos. Um dos principais desafios enfrentados ocorreu logo no início, ao subestimar a complexidade da primeira tarefa, que também representava o primeiro contato prático com a Engine e a estrutura do jogo. Essa etapa revelou a importância de reservar mais tempo para a fase inicial de familiarização e experimentação.

Houve acertos estratégicos que contribuíram significativamente para a viabilidade do projeto. A escolha por um jogo 2D e baseado em uma mecânica já conhecida (inspirada nos *Light Cycles* de Tron) foi essencial para evitar a reinvenção de conceitos e permitiu concentrar esforços na adaptação e melhoria das funcionalidades. As soluções simplificadas, aplicadas principalmente às mecânicas e à organização do código, facilitaram o entendimento das estruturas internas da Engine e possibilitaram uma evolução técnica consistente ao longo do projeto.

Os riscos de software é um problema potencial que pode afetar negativamente o cronograma, a qualidade ou o desempenho do projeto \cite{pressman2016engenharia} \cite{pmbok2017}.Ficou evidente que a ausência de um gerenciamento de riscos mais estruturado dificultou a antecipação de imprevistos, como mudanças no escopo e desafios técnicos inesperados. O planejamento inicial, ainda que básico, demonstrou ser um elemento crucial para manter o foco e organizar as entregas de forma coesa.

Durante a preparação para o projeto, foi realizado um curso introdutório de Godot e GDScript por meio da plataforma Udemy, ministrado por Davi Bandeira \cite{bandeira2024godot}. O curso abordava conceitos básicos de lógica de programação e construção de jogos simples na Engine. Embora a experiência tenha sido positiva e proporcionado segurança inicial, ao iniciar o desenvolvimento do jogo próprio, surgiu a percepção de que a dependência de tutoriais com soluções prontas limitava a autonomia na resolução de problemas. Acostumado a seguir uma "receita", foi desafiador lidar com decisões técnicas de forma independente e contextualizada.

Esse contraste reforçou uma lição importante: à medida que um desenvolvedor começa a construir um jogo por conta própria, sem se apoiar diretamente em tutoriais passo a passo, ele se depara com obstáculos mais reais, que exigem criatividade, leitura da documentação e adaptação ao funcionamento da Engine. Essa prática estimula o pensamento crítico e, sobretudo, gera confiança para iniciar projetos mais complexos e com mecânicas mais inovadoras.

Portanto, para quem estar começando, recomendo que futuros projetos priorizem um planejamento claro, escolham abordagens viáveis, considerem estratégias de mitigação de riscos desde as primeiras etapas e busquem um equilíbrio entre referências externas e a autonomia criativa no processo de desenvolvimento.


\section{Melhorias futuras}

Com o protótipo funcional concluído, diversas melhorias podem ser exploradas em uma continuidade do projeto. Entre as mais relevantes estão:

\begin{itemize}
  \item Implementação de uma inteligência artificial simples para permitir partidas solo contra o computador;
  \item Adição de suporte a partidas remotas entre jogadores por meio de conexão ponto a ponto (\textit{peer-to-peer});
  \item Inclusão de recompensas visuais ou sonoras, como efeitos de vitória, conquistas ou feedbacks de desempenho;
  \item Expansão dos menus com opções de configuração de áudio, controles e parâmetros de jogo;
  \item Criação de mapas com tamanhos e proporções ajustáveis, permitindo personalização do campo de jogo;
  \item Desenvolvimento de novos tipos de poderes especiais para ampliar a variedade de estratégias;
  \item Aprimoramento do sistema de colisão, buscando mais precisão e confiabilidade no registro de impactos;
  \item Preparação do jogo para publicação, com foco em empacotamento, identidade visual e publicação em plataformas como a Steam;
\end{itemize}

Essas propostas visam transformar o protótipo em um produto mais robusto, acessível e com maior valor de rejogabilidade, mantendo a essência competitiva e estratégica da experiência original.
