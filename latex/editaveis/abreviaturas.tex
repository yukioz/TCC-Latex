\begin{siglas}
  \item[ABNT] Associação Brasileira de Normas Técnicas
  \item[autoload] Script carregado globalmente em tempo de execução pela Godot.
  \item[CamelCase] Padrão de nomenclatura em que cada palavra inicia com letra maiúscula.
  \item[CanvasLayer] Nó da Godot usado para interface sobreposta à cena principal.
  \item[containers] Nós do tipo Control utilizados para organizar automaticamente a interface gráfica com base em regras de alinhamento, preenchimento e hierarquia.
  \item[Dr.] Doutor
  \item[FCTE] Faculdade de Ciência e Tecnologia em Engenharia
  \item[GameManager] Script responsável por gerenciar transições de cenas e resolver interações durante o jogo.
  \item[GameState] Script global (autoload) responsável por armazenar e gerenciar dados persistentes do jogo, como pontuações.
  \item[HBoxContainer] Tipo de container da Godot que organiza elementos filhos horizontalmente.
  \item[HUD] Sigla para Heads-Up Display; elementos gráficos da interface que exibem informações durante o jogo, como pontuação, tempo e status.
  \item[keybinds] Associações entre ações do jogo e teclas específicas do teclado, configuráveis pelo jogador.
  \item[lobby] Tela ou ambiente onde os jogadores realizam a seleção de personagens e configurações antes do início da partida.
  \item[low-code] Abordagem de desenvolvimento que permite criar funcionalidades com pouco ou nenhum código, geralmente por meio de interfaces visuais.
  \item[MarginContainer] Container que aplica margens internas aos seus elementos filhos, útil para criar espaçamento interno uniforme.
  \item[MDA] Mechanics, Dynamics, Aesthetics (Mecânicas, Dinâmicas e Estética).
  \item[Node2D] Tipo de nó base para objetos 2D na Godot Engine.
  \item[p.] Página
  \item[\texttt{physics\_process}] Função de script da Godot chamada em intervalos fixos, utilizada para lógica relacionada à física e movimentação contínua.
  \item[Prof.] Professor
  \item[S.d] Sem data
  \item[\texttt{snake\_case}] Padrão de nomenclatura com palavras minúsculas separadas por sublinhado.
  \item[TCC] Trabalho de Conclusão de Curso.
  \item[trail] Rastro visual deixado pelos jogadores durante a movimentação, geralmente utilizado como elemento de colisão e estratégia no gameplay.
  \item[VBoxContainer] Tipo de container da Godot que organiza elementos filhos verticalmente.
\end{siglas}
